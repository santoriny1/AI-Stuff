\documentclass[12pt]{article}

\usepackage[english,spanish]{babel}
\usepackage[utf8]{inputenc} 
\usepackage{cite}
\usepackage{array}
\usepackage{enumerate}
\usepackage{stix}
\usepackage{graphicx}
\usepackage{ragged2e}
\usepackage[a4paper,top=3cm,bottom=2cm,left=3cm,right=3cm,marginparwidth=1.75cm]{geometry}
\usepackage{amsmath}
\usepackage[colorinlistoftodos]{todonotes}
\usepackage[colorlinks=true, allcolors=blue]{hyperref}
\usepackage{mathpazo}

\begin{document}

\begin{titlepage}
	\newcommand{\HRule}{\rule{\linewidth}{0.5mm}} 
	
	\center 
	
	
	
	\textsc{\LARGE Instituto Tecnológico de Morelia }\\[1.5cm] 
    
    \includegraphics[width=0.2\textwidth]{descarga.jpg}\\[1cm] 
	
	\textsc{\Large Ing. en Sistemas Computacionales}\\[0.5cm] 
	
	\textsc{\large Ensayo }\\[0.5cm] 
	
	
	
	\HRule\\[0.4cm]
	
	{\huge\bfseries Introducción a la Inteligencia Artificial}\\[0.4cm]
	
	\HRule\\[1.5cm]
	
	
	
	\begin{minipage}{0.4\textwidth}
		\begin{flushleft}
			\large
			\textit{Autor}\\
		    \textsc{David Zavala Moreno} 
		\end{flushleft}
	\end{minipage}
	~
	\begin{minipage}{0.5\textwidth}
		\begin{flushright}
			\large
			\textit{Profesor}\\
			\textsc{J. Eduardo Alcaraz Chávez}
		\end{flushright}
	\end{minipage}
  
\end{titlepage}

\newpage
\newpage
\tableofcontents 
\newpage

\section{Introducción}

A lo largo de este trabajo se analizará el pensar y si es que las máquinas puedan llegar a lograr esta habilidad, también se profundizará en las consecuencias y complicaciones que existirían si estas lograrán hacerlo. Mucho antes de que si quiera existieran las máquinas, el ser humano ya había tenido el dilema sobre como es que nuestra mente trabaja y si es posible que cualquier otro objeto, ser o cosa tenía la capacidad del pensamiento.\\

Se ha logrado un gran desarrollo en cuanto al entendimiento, habilidades, diseños y componentes de agentes racionales, pero realmente aún no estamos seguros de su progreso pueda actuar siempre de manera eficiente ante distintas situaciones o diferentes entornos.\\

Existen dos hipótesis sobre como la Inteligencia Artificial puede funcionar en las máquinas. La primera y la más aceptada nos dice que las máquinas solo pueden simular la inteligencia, esto quiere decir que pueden actuar inteligentemente pero no están pensando realmente. Mientras tanto la segunda hipótesis nos dice que las máquinas pueden llegar a pensar y tener inteligencia real, lo cual es una idea contraría a nuestra primera hipótesis.
Sin embargo, cualesquiera que sea la hipótesis aceptada, lo único que actualmente importa es que la Inteligencia Artificial resuelva los problemas para los cuales fue creada.

\section{Fundamentos filosóficos}

Durante mucho tiempo se ha pensado que la Inteligencia Artificial es imposible ya que una máquina no podrá acercarse al pensamiento racional propio nunca. Muchas personas piensan que esto depende de como se interprete y defina el pensamiento ya que como sabemos existe un debate filosófico desde la antigüedad sobre este tema, el cual en tiempos modernos aún no esta bien definido.\\

Alan Turing un grande y famoso genio dentro del mundo de la computación planteó la posibilidad de evaluar a las máquinas mediante un test de inteligencia y no cuestionarnos si es posible que tienen la capacidad de pensar. El test es bastante sencillo, solamente se tienes que sostener una conversación entre una máquina y un humano durante cinco minutos y si el humano no se da cuenta que esta hablando con una máquina al menos durante el setenta por ciento de la prueba, la máquina aprobará.
Aún así, en tiempos actuales existen máquinas que han logrado engañar a un humano por más de cinco minutos (cabe mencionar que solo a personas que ignorantes sobre el tema). Sin embargo a jueces con conocimiento, las máquinas aún no logran acercarse al treinta por ciento requerido.\\

Turing también mencionó que existen habilidades o características que una máquina jamás lograría tales como:
\begin{itemize}
    \item \textbf{Capacidad de:} Tener una diversidad de comportamientos como el hombre, tener gustos o preferencias, empirismo, creatividad.
    \item \textbf{Capacidad matemática:} Ciertas cuestiones matemáticas no pueden ser respondidas por sistemas formales, los cuales son los que pueden realizar los computadores.
    \item \textbf{Informalidad:} El comportamiento humano es tan complejo que es imposible adaptarlo a reglas, por lo que no puede ser traducido o interpretado por una máquina.
\end{itemize}

Mientras que una máquina no sea consiente de sus acciones no se podrá asumir que es un ser pensante. Aunado a esto esta la cuestión de los sentimientos, ya que una máquina no puede tener emociones y si se pudieran llegar a programar estas, entonces, no se tendría la certeza de que son emociones verdaderas.\\

Hasta ahora en lo único que se ha centrado la humanidad es en desarrollar Inteligencia Artificial pero sin cuestionarse si debería o no hacerlo. Aún no se saben las consecuencias que tendría para nosotros, recordemos que muchos de los inventos más recientes han tenido efectos negativos sobre la humanidad es por esto que se deberían analizar cuidadosamente las implicaciones éticas y morales que representa el desarrollo de la Inteligencia Artificial. 

\section{IA: presente y futuro}

Una de las principales debilidades de la Inteligencia Artificial actual es que para funcionar se le debe proporcionar entradas e interpretar las salidas que produce, careciendo entonces de razonamiento y planificación de alto nivel, es como un niño que sabe leer pero no entiende qué leyó ni por qué lo leyó.
Sin embargo, pese a que las máquinas aún no pueden construir representaciones nuevas de abstracción en las entradas, es posible mediante una jerarquía de aprendizaje introducir acciones simples a la construcción de tareas de alto nivel por medio de las bases de conocimiento jerárquico.\\

Ahora que se tienen avances y oportunidades de progresar en la Inteligencia Artificial deberíamos poder predecir cual será nuestro siguiente paso en la búsqueda de la creación de conocimiento, por un lado podríamos seguir por la misma rama e intentar lograr avances donde no sabemos si existe posibilidad de avanzar o empezar a retroceder y replantear la idea que se ha tenido durante todos estos años a cerca de la Inteligencia Artificial.\\

Pero y si se pudiera desarrollar la inteligencia Artificial que tanto se ha buscado cual sería el siguiente paso de los investigadores dentro de este campo, queda claro que una computadora inteligente es más útil que una que no lo es. Sin embargo cuántas serían necesarias y para qué serían necesarias, es en este punto en donde volvemos a retomar una idea anterior que era si un logró como el desarrollo de un Inteligencia Artificial sería bueno o malo para la humanidad, por ejemplo, la energía nuclear tenía un buen propósito pero ahora sabemos por muchas malas experiencias lo contraproducente que fue su creación.
Nuevamente caemos en los dilemas éticos y morales ya mencionados.

\section{Fundamentos matemáticos}

Para el desarrollo de la Inteligencia Artificial es necesaria la comparación de algoritmos para determinar cual es el que mejor desempeña la tarea que se requiere optimizando la mayor cantidad de recursos. Existen diferentes criterios en los cuales se puede medir esto: 
\begin{itemize}
    \item Tiempo
    \item Compilador
    \item Lenguaje
    \item Rendimiento
\end{itemize}

Otro método de análisis es el de complejidad. Sin embargo, este analiza problemas y no algoritmos, se basa en encontrar problemas que se puedan resolver en un tiempo polinomial y problemas que no, sin importar el algoritmo que se utilice.  

\section{Conclusiones}

Como se menciona al principio de este ensayo el pensar y su interpretación han sido uno de los dilemas filosóficos más importantes y estudiados a lo largo de la historia, Sin embargo, aún no se tiene una definición concreta y esto afecta directamente a las hipótesis sobre la Inteligencia Artificial que se tienen actualmente pero, mientras se puedan resolver los problemas para los que se intenta desarrollar la Inteligencia Artificial deberíamos cuestionarnos si realmente lo que nos importa como humanidad es crear a otro ente pensante o simplemente seguir satisfaciendo nuestras necesidades a través de el sin importar que simule inteligencia.\\

En cuanto al desarrollo actual de la Inteligencia Artificial aún no sabemos si estamos siguiendo el camino correcto o si es necesario un replanteamiento de todo lo aprendido hasta ahora sobre la Inteligencia Artificial, lo que si sabemos es que aunque sea en una dirección equivocada existe un camino que seguir y aunque sea pequeño, sabemos que hay mucho por descubrir.

\end{document}